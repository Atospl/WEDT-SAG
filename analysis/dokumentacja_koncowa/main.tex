\documentclass[a4paper]{article}

%% Language and font encodings
\usepackage{polski}
\usepackage{multirow}
\usepackage{longtable}
\usepackage{amsmath}
\usepackage[utf8]{inputenc}
\usepackage{float}
\pagenumbering{gobble}
\usepackage[T1]{fontenc}

%% Sets page size and margins
\usepackage[a4paper,top=3cm,bottom=2cm,left=3cm,right=3cm,marginparwidth=1.75cm]{geometry}

%% Useful packages

\usepackage{amsmath}
\usepackage{graphicx}
\usepackage[colorinlistoftodos]{todonotes}
\usepackage[colorlinks=true, allcolors=blue]{hyperref}
\usepackage{hyperref}

\title{\Large{Dokumentacja końcowa projektu} \\
Projekt łączony z przedmiotów \textit{Systemy Agentowe} oraz \textit{Wstęp do eksploracji danych tekstowych w sieci WWW}}
\author{Andrzej Dawidziuk, Tomasz Kogowski, Rafał Okuniewski}
\begin{document}
\maketitle

\section{Wstęp}

\subsection{Zadanie projektowe}
\par Zadanie projektowe polega na wyszukiwaniu tematycznych wiadomości w Internecie oraz ich analizie semantycznej pod kątem podobieństwa i tematyki. Na potrzeby projektu analizowane będą portale informacyjne udostępniające treści w języku angielskim.
\par Celem projektu było stworzenie rozproszonego systemu agentowego, który umożliwi zapisywanie oraz wektoryzację artykułów z różnych internetowych serwisów informacyjnych, oraz umożliwi wskazanie artykułów o podobnej treści, odnoszących się do podobnych tematów, wydarzeń, wypowiedzi.

\subsection{Wykorzystane technologie}

\par W celu realizacji systemu agentowego wykorzystano powszechnie stosowany framework Akka\cite{Akka}. Umożliwia on stworzenie funkcjonalnego systemu agentowego w dwóch językach - Scala oraz Java. Zespół podjął decyzję o implementacji w języku Scala.
\par Warstwę bazodanową, w której przechowywane były informacje o pobranych ze stron informacyjnych artykułach, stanowiła baza danych Postgres.
\par W celu stworzenia modułu odpowiedzialnego za wektoryzację oraz analizę podobieństwa artykułów wykorzystano język Python oraz biblioteki NLTK\cite{nltk} i GenSim\cite{gensim}.
\par Ze względu na konieczność integracji modułu agentowego z modułem analitycznym, zdecydowano o integracji poprzez wykonywanie zapytań do serwera HTTP w którym osadzony był moduł analityczny. Serwer ten został wykonany z wykorzystaniem biblioteki flask\cite{flask}.

\section{System agentowy}
%%%% OPIS NARZĘDZI - plusy i minusy

\subsection{Język Scala}

\subsection{Framework Akka}

\section{Przetwarzanie języka naturalnego}
%%%% OPIS NARZĘDZI - plusy i minusy

Do określania podobieństwa artykułów użyta została technika
doc2vec. Przy jej użyciu możliwe jest wygenerowanie dla każdego z
artykułów wektora liczb rzeczywistych, osadzającego cały dokument w
przestrzeni wektorowej. Wektory dokumentów podobnych do siebie powinny
leżeć w tej przestrzeni blisko siebie.


\subsection{Doc2Vec}

Nazwa doc2vec\cite{doc2vec} (lub paragraph2vec) jest wspólną nazwą używaną do
określenia dwóch różnych algorytmów:

\begin{itemize}
\item Distributed memory model
\item Distributed bag of words
\end{itemize}

Algorytmy te umożliwiają wyznaczenie wektorowej reprezentacji dokumentu za pomocą \textit{uczenia bez nadzoru.} Dzięki temu możliwe jest wytrenowanie modelu doc2vec, służącego do wyznaczania wektorów dla jeszcze nie widzianych artykułów na nieoznaczonym korpusie.

Doc2vec jest rozszerzeniem techniki word2vec służącej do wyznaczania
wektorowych reprezentacji pojedynczych słów na podstawie kontekstu w
jakim występują w korpusie źródłowym. Tak więc do jego zastosowania
potrzebne są reprezentacje słów (przygotowane wcześniej lub wyznaczone
podczas treningu modelu doc2vec). Oczywistym ograniczeniem algorytmu
jest to, że lista słów rozpoznawanych przez model ograniczona jest do
słów występujących w korpusie treningowym. Tak więc korpus treningowy
powinien być jak najbardziej zbliżony do rzeczywistych artykułów.

\subsubsection{Podobieństwo wektorów}

Podobieństwo wektorów wyznaczonych przez model doc2vec określane jest
przez wyznaczenie iloczynu skalarnego ich wersorów:

\begin{equation}
  S(\mathbf{u},\mathbf{v}) = \frac{\mathbf{u}}{||\mathbf{u}||} \cdot \frac{\mathbf{v}}{||\mathbf{v}||}
\end{equation}

\subsubsection{Modele}
\label{ssec:modele}

Do klasyfikacji artykułów zostały użyte trzy różne modele, trenowane
na dwóch różnych korpusach:

\begin{itemize}
\item Routers -- zbiór artykułów Reutersa, dostępny w bibliotece
  NLTK. W skład korpusu, oprócz artykułów wchodzą również krótkie
  depesze, składające się głównie ze skrótów nazw spółek i kwot
  transakcji przez nie zawieranych,
\item Rzeczywisty -- zbiór ok. 400 artykułów uzyskanych przez system w
  trakcie jego działania.
\end{itemize}

Wytrenowane i ocenione zostały trzy modele, ich wyniki znajdują się w sekcji \ref{sec:wyniki}:

\begin{itemize}
\item reuters -- model trenowany na korpusie Reuters,
\item reuters+rzeczywisty -- model trenowany na obu korpusach,
\item rzeczywisty -- model trenowany na korpusie rzeczywistych wiadomości.
\end{itemize}

\paragraph{Preprocessing}

Przed treningiem korpusy poddane zostały preprocessingowi: usunięte i
zastąpione spacją zostały wszystkie symbole nie będące literami lub
cyframi.


\subsection{GenSim}

Do wytrenowania modeli doc2vec użyty został framework Gensim i klasa
\texttt{Doc2Vec} z modułu \texttt{gensim.models.doc2vec}. Umożliwia
ona trenowanie modeli doc2vec, jak również wyznaczanie wektorów dla
niewidzianych wcześniej dokumentów.

Pewnym utrudnieniem w jej użyciu była bardzo ogólnikowa dokumentacja,
jak również brak niektórych oczywistych funkcji, np. policzenie
podobieństwa wektorów przy użyciu dostępnej w bibliotece metody
\texttt{infer\_vector} możliwe jest tylko dla wektorów reprezentujących
dokumenty ze zbioru treningowego. Tak więc ta funkcjonalność musiała
zostać zaimplementowana ręcznie.

Oprócz braku dokumentacji, wynikającego z trwającego jeszcze rozwoju
frameworku, Gensim nie sprawił większych kłopotów. Zarówno prędkość
treningu, jak i wyznaczania reprezentacji wektorowych jest
wystarczająca dla potrzeb projektu. Niestety, funkcje dostępne w
klasie \texttt{Doc2vec} nie są przystosowane do działania na GPU,
którego użycie znacząco przyspiesza trening sieci neuronowych.



\section{Badania}

\subsection{Opis zbioru danych}

\subsubsection{Zbiór rzeczywisty}

\subsubsection{Zbiór sztuczny}

Zbiór sztuczny został stworzony ręcznie przez twórców systemu i zawierał dwa rodzaje
artykułów. 
\begin{itemize}
\item artykuły identyczne - o tej samej tematyce,
\item artykuły zbliżone - o zbliżonej tematyce
\end{itemize}

Utworzono 3 grupy artykułów identycznych, każdy po 10 artykułów.
\begin{itemize}
\item sztorm w Holandii oraz Niemczech,
\item wyrzut cząsteczek z czarnej dziury,
\item podniesienie ceny serwisu Amazon Prime,
\end{itemize}

Artykułów o zbliżonej tematyce również powstały 3 grupy, również po 10 artykułów.
\begin{itemize}
\item Putin,
\item Chelsea,
\item meltdown
\end{itemize}

\subsection{Znajdowanie artykułów podobnych jako zagadnienie klasyfikacji}

\subsection{Rezultaty}

\section{Podsumowanie}

\bibliographystyle{unsrt}
\bibliography{bibliography}

\end{document}